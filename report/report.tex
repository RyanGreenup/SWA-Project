% Created 2020-04-26 Sun 15:48
% Intended LaTeX compiler: pdflatex
\documentclass[11pt]{article}
\usepackage[utf8]{inputenc}
\usepackage[T1]{fontenc}
\usepackage{graphicx}
\usepackage{grffile}
\usepackage{longtable}
\usepackage{wrapfig}
\usepackage{rotating}
\usepackage[normalem]{ulem}
\usepackage{amsmath}
\usepackage{textcomp}
\usepackage{amssymb}
\usepackage{capt-of}
\usepackage{hyperref}
\usepackage{minted}
\usepackage{/home/ryan/Dropbox/profiles/Templates/LaTeX/ScreenStyle}
\author{Ryan Greenup}
\date{\today}
\title{Analysing Twitter for Ubisoft}
\hypersetup{
 pdfauthor={Ryan Greenup},
 pdftitle={Analysing Twitter for Ubisoft},
 pdfkeywords={},
 pdfsubject={},
 pdfcreator={Emacs 27.0.91 (Org mode 9.4)}, 
 pdflang={English}}
\begin{document}

\maketitle
\tableofcontents



\section{8.1 Analysing the Relationship Between Friends and Followers for Twitter Users}
\label{sec:org41527e8}
\subsection{8.1.1 Retrieve the posts from Twitter}
\label{sec:org8520f5d}
relevant posts can be retrieved from twitter by utilising the \texttt{rtweet} package, packages can be loaded for use in \textbf{\textbf{\uline{R}}} thusly:

\begin{listing}[htbp]
\begin{minted}[]{r}
# Load Packages -----------------------------------------------------------
setwd("~/Dropbox/Notes/DataSci/Social_Web_Analytics/SWA-Project/scripts/")

if (require("pacman")) {
  library(pacman)
} else{
  install.packages("pacman")
  library(pacman)
}

pacman::p_load(xts, sp, gstat, ggplot2, rmarkdown, reshape2,
               ggmap, parallel, dplyr, plotly, tidyverse,
               reticulate, UsingR, Rmpfr, swirl, corrplot,
               gridExtra, mise, latex2exp, tree, rpart,
               lattice, coin, primes, epitools, maps, clipr,
               ggmap, twitteR, ROAuth, tm, rtweet, base64enc,
               httpuv, SnowballC, RColorBrewer, wordcloud,
               ggwordcloud)
\end{minted}
\caption{\label{org836710b}Load the Packages for \textbf{\textbf{\emph{R}}}}
\end{listing}

The \texttt{rtweet} API will search for tweets that contain all the words of a query
regardless of uppercase or lowercase usage \cite{kearney2019}.

In order to leverage the \emph{Twitter} API it is necessary to use tokens provided through a \emph{Twitter} developer account:

\begin{listing}[htbp]
\begin{minted}[]{r}
# Set up Tokens ===========================================================

options(RCurlOptions = list(
  verbose = FALSE,
  capath = system.file("CurlSSL", "cacert.pem", package = "RCurl"),
  ssl.verifypeer = FALSE
))

setup_twitter_oauth(
  consumer_key = "*************************",
  consumer_secret = "**************************************************",
  access_token = "**************************************************",
  access_secret = "*********************************************"
)

# rtweet ==================================================================
tk <-    rtweet::create_token(
  app = "SWA",
  consumer_key    = "*************************",
  consumer_secret = "**************************************************",
  access_token    = "**************************************************",
  access_secret   = "*********************************************",
  set_renv        = FALSE
\end{minted}
\caption{\label{org3e7a889}Import the twitter tokens (redacted)}
\end{listing}

and hence all tweets containing a mention of \emph{Ubisoft} can be returned and saved to disk as shown in listing \ref{orge140b87}:

\begin{listing}[htbp]
\begin{minted}[]{r}
 n <- 1000
 tweets.company <- search_tweets(q = 'ubisoft', n = n, token = tk,
                                 include_rts = FALSE)
 save(tweets.company[,], file = "resources/Download_1.Rdata")
\end{minted}
\caption{\label{orge140b87}Save the Tweets to the HDD as an \texttt{rdata} file}
\end{listing}



\section{References}
\label{sec:org1995b96}
\label{orgeb6e6ae}
\bibliography{references}

\label{org7d815aa}
 \bibliographystyle{unsrt}
\end{document}
