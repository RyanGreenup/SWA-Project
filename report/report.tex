% Created 2020-04-26 Sun 13:58
% Intended LaTeX compiler: pdflatex
\documentclass[11pt]{article}
\usepackage[utf8]{inputenc}
\usepackage[T1]{fontenc}
\usepackage{graphicx}
\usepackage{grffile}
\usepackage{longtable}
\usepackage{wrapfig}
\usepackage{rotating}
\usepackage[normalem]{ulem}
\usepackage{amsmath}
\usepackage{textcomp}
\usepackage{amssymb}
\usepackage{capt-of}
\usepackage{hyperref}
\usepackage{/home/ryan/Dropbox/profiles/Templates/LaTeX/ScreenStyle}
\author{Ryan Greenup}
\date{\today}
\title{Analysing Twitter for SquareEnix}
\hypersetup{
 pdfauthor={Ryan Greenup},
 pdftitle={Analysing Twitter for SquareEnix},
 pdfkeywords={},
 pdfsubject={},
 pdfcreator={Emacs 27.0.91 (Org mode 9.4)}, 
 pdflang={English}}
\begin{document}

\maketitle
\tableofcontents



\section{8.1 Analysing the Relationship Between Friends and Followers for Twitter Users}
\label{sec:org9c03e62}
\subsection{8.1.1 Retrieve the posts from Twitter}
\label{sec:orgd718371}
relevant posts can be retrieved from twitter by leveraging the \texttt{rtweet} package. Unfourtunately \emph{SquareEnix} is a compound word and so it will be necessary to return results to match \texttt{SquareEnix} or \texttt{Square Enix}, this can be acheived by specifying the regex \texttt{.*square*}:

The \texttt{rtweet} API will search for tweets that contain all the words of a query whether or not they are compounded together or seperated by \emph{whitespace} characters \cite{kearney2019}.


\section{References}
\label{sec:orgb0d8e9b}
\end{document}
